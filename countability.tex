\section{Countability of Sets}
	\begin{description}
		\item[Task:] Understand what it means for a set to be countable, countably infinite and uncountably infinite. Show that the set of all languages over a finite alphabet is uncountably infinite, wheras the set ofa ll regular languages over a finite alphabet is countably infinite.
	\end{description}

	We want to understand sizes of sets.
	In the unit on functions last term, when we looked at functions defined on finite sets, we wrote down a set A with m elements as $A = \{a_1, \dots, a_n\}$.
	This notation \underline{mimics the process of counting:} $a_1$ is the first element of $A$, $a_2$ is the second element of $A$, and so on up to $a_n$ is the $n^th$ element of $A$.
	In other words, another way of saying $A$ in a set of $n$ elements is that there exists a bijective function $f: A \rightarrow \{1, 2, \dots, n\}$.

	\begin{description}
		\item[Definition:] A set $A$ has $n$ elements $\leftrightarrow \exists f: A \rightarrow J_n$ a bijection.
		\item[NB:] This definition works $\forall n \geq 1, n \in \mathbb{N}^\ast$
		\item[Notation:] $\exists f: A \rightarrow J_n$ a bijection is denoted as $A \sim J_n$.
			More generally, $A \sim B$ means $\exists f: A \rightarrow B$ a bijection, and it is a relation on sets.
			In fact, it is an equivalence relation (check!). $[J_n]$ is the equivalence class of all sets $A$ of size $n$, i.e. $\#(A) = n$.
	\end{description}

	\begin{description}
		\item[Definition:] A set $A$ is \underline{finite} if $A \sim J_n$ for some $n \in \mathbb{N}^\ast$ or $A = \emptyset$.
		\item[Definition:] A set $A$ is \underline{infinite} if $A$ is not finite.
		\item[Examples:] $\mathbb{N, Q, R,}$ etc.
	\end{description}

	To understand sizes of infinite sets, generalize the construction above. Let $J = \mathbb{N}^\ast = \{1, 2, \dots \}$

	\begin{description}
		\item[Definition:] A set $A$ is \underline{uncountably infinite} if $A$ is neither finite nor countably infinite.
			In fact, we often treat together the cases $A$ is finite or $A$ is countably infinite since in both of these cases the counting mechanism that is so familiar to us works.
			Therefore, we have the following definition:
		\item[Definition:] A set $A$ is \underline{countable} if $A$ is finite ($A \sim J_n$ or $A = \emptyset$) or $A$ is countably infinite ($A \sim J$).
	\end{description}
	There is a difference in the terminology regarding countability between CS sources (textbooks, articles, etc.) and maths sources.
	This is the dictionary:
	\centerline{
		\begin{tabular}{|c|c|}
			\hline
			CS & Maths	\\
			\hline
			countable & at most countable	\\
			\hline
			countably infinite & countable \\
			\hline
			uncountably infinite & uncountable \\
			\hline
		\end{tabular}
	}
	It's best to double check which terminology a source is using.

	\begin{description}
		\item[Goal:] Characterize $[\mathbb{N}]$, the equivalence class of countably infinite set, and $[\mathbb{R}]$, the equivalence class of uncountably infinite sets the same size as $\mathbb{R}$.
		\item[Bad news:] Both $[\mathbb{N}]$ and $[\mathbb{R}]$ consist of ininite sets.
		\item[Good news:] We only care about these two equivalence classes.
		\item[NB:] There are uncountably infinite sets of size bigger than $[\mathbb{R}]$ that can be obtained from the power set construction, but it is unlikely you will see them in your CS coursework.
	\end{description}

	To characterize $[\mathbb{N}]$, we need to recall the notion of a sequence:

	\begin{description}
		\item[Definition:] A \underline{sequence} is a set of elements $\{x_1, x_2, \dots\}$ indexed by $J$, i.e. $\exists f: \rightarrow \{x_1, x_2, \dots\}$ s.t. $f(n) = x_n \forall n \in J$.
	\end{description}

	Recall that sequences and their limits are used to define various notions in calculus (differentiation, integration, etc.).
	Also, calculators use sequences in order to compute with various rational and irrational numbers.
